\documentclass[11pt]{article}
\usepackage[utf8]{inputenc}
\usepackage{amsmath}
\usepackage{graphicx}
\usepackage{xcolor}
\usepackage{listings}
\usepackage{caption}
\usepackage{geometry}
\usepackage{float} % To use [H] for figures
\geometry{margin=1in}

\title{Foundation of Data Engineering}
\author{}
\date{}

\definecolor{codegray}{rgb}{0.5,0.5,0.5}
\definecolor{codepurple}{rgb}{0.58,0,0.82}
\definecolor{backcolour}{rgb}{0.95,0.95,0.92}

\lstdefinestyle{pythonstyle}{
    backgroundcolor=\color{backcolour},   
    commentstyle=\color{codegray},
    keywordstyle=\color{blue},
    numberstyle=\tiny\color{codegray},
    stringstyle=\color{codepurple},
    basicstyle=\ttfamily\footnotesize,
    breaklines=true,                 
    captionpos=b,                    
    keepspaces=true,                 
    numbers=left,                    
    numbersep=5pt,                  
    showspaces=false,                
    showstringspaces=false,
    showtabs=false,                  
    tabsize=2
}

\lstset{style=pythonstyle}

\begin{document}

\maketitle

\section*{Functions to Plot}

\section{Activation Functions}

\subsection{Sigmoid Function}

\begin{lstlisting}[language=Python, caption={Sigmoid function and its derivative}]
import numpy as np
import matplotlib.pyplot as plt

def sigmoid(x):
    return 1 / (1 + np.exp(-x))

def sigmoid_derivative(x):
    s = sigmoid(x)
    return s * (1 - s)

x = np.linspace(-10, 10, 1000)
y_sigmoid = sigmoid(x)
y_derivative = sigmoid_derivative(x)

plt.figure(figsize=(10, 5))
plt.subplot(1, 2, 1)
plt.plot(x, y_sigmoid, label="Sigmoid", color='green')
plt.title("Sigmoid Function")
plt.xlabel("x")
plt.ylabel("sigma(x)")
plt.grid(True)
plt.legend()

plt.subplot(1, 2, 2)
plt.plot(x, y_derivative, label="Sigmoid Derivative", color='orange')
plt.title("Derivative of Sigmoid")
plt.xlabel("x")
plt.ylabel("d(sigma)/dx")
plt.grid(True)
plt.legend()

plt.tight_layout()
plt.show()
\end{lstlisting}

\noindent\textbf{Output:}
\begin{figure}[H]
    \centering
    \includegraphics[width=0.8\textwidth]{sigmoid.png}
    \caption{Sigmoid function and its derivative}
\end{figure}

\subsection{ReLU Function}

\begin{lstlisting}[language=Python, caption={ReLU function and its derivative}]
import numpy as np
import matplotlib.pyplot as plt

def relu(x):
    return np.maximum(0, x)

def relu_derivative(x):
    return np.where(x > 0, 1, 0)

x = np.linspace(-10, 10, 1000)
y_relu = relu(x)
y_derivative = relu_derivative(x)

plt.figure(figsize=(10, 5))
plt.subplot(1, 2, 1)
plt.plot(x, y_relu, label="ReLU", color='blue')
plt.title("ReLU Function")
plt.xlabel("x")
plt.ylabel("ReLU(x)")
plt.grid(True)
plt.legend()

plt.subplot(1, 2, 2)
plt.plot(x, y_derivative, label="ReLU Derivative", color='red')
plt.title("Derivative of ReLU")
plt.xlabel("x")
plt.ylabel("d(ReLU)/dx")
plt.grid(True)
plt.legend()

plt.tight_layout()
plt.show()
\end{lstlisting}

\noindent\textbf{Output:}
\begin{figure}[H]
    \centering
    \includegraphics[width=0.8\textwidth]{Relu.png}
    \caption{ReLU function and its derivative}
\end{figure}

\subsection{Tanh Function}

\begin{lstlisting}[language=Python, caption={Tanh function and its derivative}]
import numpy as np
import matplotlib.pyplot as plt

def tanh(x):
    return np.tanh(x)

def tanh_derivative(x):
    return 1 - np.tanh(x)**2

x = np.linspace(-10, 10, 1000)
y_tanh = tanh(x)
y_derivative = tanh_derivative(x)

plt.figure(figsize=(10, 5))
plt.subplot(1, 2, 1)
plt.plot(x, y_tanh, label="Tanh", color='purple')
plt.title("Tanh Function")
plt.xlabel("x")
plt.ylabel("tanh(x)")
plt.grid(True)
plt.legend()

plt.subplot(1, 2, 2)
plt.plot(x, y_derivative, label="Tanh Derivative", color='brown')
plt.title("Derivative of Tanh")
plt.xlabel("x")
plt.ylabel("d(tanh)/dx")
plt.grid(True)
plt.legend()

plt.tight_layout()
plt.show()
\end{lstlisting}

\noindent\textbf{Output:}
\begin{figure}[H]
    \centering
    \includegraphics[width=0.8\textwidth]{Tanh.png}
    \caption{Tanh function and its derivative}
\end{figure}

\section{Basic Functions}

\subsection{$f(x) = e^{-x}$}

\begin{lstlisting}[language=Python, caption={$f(x) = e^{-x}$ and its derivative}]
import numpy as np
import matplotlib.pyplot as plt

def f(x):
    return np.exp(-x)

def f_derivative(x):
    return -np.exp(-x)

x = np.linspace(-5, 5, 500)
y = f(x)
y_prime = f_derivative(x)

plt.figure(figsize=(10, 5))
plt.plot(x, y, label=r'$f(x) = e^{-x}$', color='blue')
plt.plot(x, y_prime, label=r"$f'(x) = -e^{-x}$", color='red', linestyle='--')
plt.title("Function and Derivative: $f(x) = e^{-x}$")
plt.xlabel("x")
plt.ylabel("y")
plt.axhline(0, color='black', linewidth=0.5)
plt.axvline(0, color='black', linewidth=0.5)
plt.grid(True)
plt.legend()
plt.tight_layout()
plt.show()
\end{lstlisting}

\noindent\textbf{Output:}
\begin{figure}[H]
    \centering
    \includegraphics[width=0.8\textwidth]{Fig1.png}
    \caption{$f(x) = e^{-x}$ function and its derivative}
\end{figure}

\subsection{$f(x) = e^{-|x|}$}

\begin{lstlisting}[language=Python, caption={$f(x) = e^{-|x|}$ and its derivative}]
import numpy as np
import matplotlib.pyplot as plt

def f(x):
    return np.exp(-np.abs(x))

def f_derivative(x):
    return np.where(x > 0, -np.exp(-x), np.where(x < 0, np.exp(x), 0))

x = np.linspace(-5, 5, 1000)
y = f(x)
y_prime = f_derivative(x)

plt.figure(figsize=(10, 5))
plt.plot(x, y, label=r'$f(x) = e^{-|x|}$', color='blue')
plt.plot(x, y_prime, label=r"$f'(x)$", color='orange', linestyle='--')
plt.title("Function and Derivative: $f(x) = e^{-|x|}$")
plt.xlabel("x")
plt.ylabel("y")
plt.grid(True)
plt.axhline(0, color='black', linewidth=0.5)
plt.axvline(0, color='black', linewidth=0.5)
plt.legend()
plt.tight_layout()
plt.show()
\end{lstlisting}

\noindent\textbf{Output:}
\begin{figure}[H]
    \centering
    \includegraphics[width=0.8\textwidth]{Fig2.png}
    \caption{$f(x) = e^{-|x|}$ function and its derivative}
\end{figure}

\subsection{$f(x) = e^{-x^2}$}

\begin{lstlisting}[language=Python, caption={$f(x) = e^{-x^2}$ and its derivative}]
import numpy as np
import matplotlib.pyplot as plt

def f(x):
    return np.exp(-x**2)

def f_derivative(x):
    return -2 * x * np.exp(-x**2)

x = np.linspace(-3, 3, 500)
y = f(x)
y_prime = f_derivative(x)

plt.figure(figsize=(10, 5))
plt.plot(x, y, label=r'$f(x) = e^{-x^2}$', color='blue')
plt.plot(x, y_prime, label=r"$f'(x) = -2x e^{-x^2}$", color='red', linestyle='--')
plt.title("Function and Derivative: $f(x) = e^{-x^2}$")
plt.xlabel("x")
plt.ylabel("y")
plt.grid(True)
plt.axhline(0, color='black', linewidth=0.5)
plt.axvline(0, color='black', linewidth=0.5)
plt.legend()
plt.tight_layout()
plt.show()
\end{lstlisting}

\noindent\textbf{Output:}
\begin{figure}[H]
    \centering
    \includegraphics[width=0.8\textwidth]{Fig3.png}
    \caption{$f(x) = e^{-x^2}$ function and its derivative}
\end{figure}

\vspace{2em}
\hrule
\vspace{1em}

\begin{center}
\textbf{Submitted By} \\
\vspace{0.5em}
Name: \textbf{Sarvesh Adithya J} \\
Registration Number: \textbf{2512004}
\end{center}

\end{document}
